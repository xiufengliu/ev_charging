% This LaTeX document needs to be compiled with XeLaTeX.
\documentclass[10pt]{article}
\usepackage[utf8]{inputenc}
\usepackage{ucharclasses}
\usepackage{hyperref}
\hypersetup{colorlinks=true, linkcolor=blue, filecolor=magenta, urlcolor=cyan,}
\urlstyle{same}
\usepackage{amsmath}
\usepackage{amsfonts}
\usepackage{amssymb}
\usepackage[version=4]{mhchem}
\usepackage{stmaryrd}
\usepackage{graphicx}
\usepackage[export]{adjustbox}
\graphicspath{ {./images/} }
\usepackage{polyglossia}
\usepackage{fontspec}
\usepackage{eurosym}
\usepackage{newunicodechar}
\setmainlanguage{english}
\setotherlanguages{hindi}
\IfFontExistsTF{Noto Serif Devanagari}
{\newfontfamily\hindifont{Noto Serif Devanagari}}
{\IfFontExistsTF{Kohinoor Devanagari}
  {\newfontfamily\hindifont{Kohinoor Devanagari}}
  {\IfFontExistsTF{Devanagari MT}
    {\newfontfamily\hindifont{Devanagari MT}}
    {\IfFontExistsTF{Lohit Devanagari}
      {\newfontfamily\hindifont{Lohit Devanagari}}
      {\IfFontExistsTF{FreeSerif}
        {\newfontfamily\hindifont{FreeSerif}}
        {\newfontfamily\hindifont{Arial Unicode MS}}
}}}}
\IfFontExistsTF{CMU Serif}
{\newfontfamily\lgcfont{CMU Serif}}
{\IfFontExistsTF{DejaVu Sans}
  {\newfontfamily\lgcfont{DejaVu Sans}}
  {\newfontfamily\lgcfont{Georgia}}
}
\setDefaultTransitions{\lgcfont}{}
\setTransitionsForDevanagari{\hindifont}{\rmfamily}

\title{Promoting electric vehicles: Reducing charging inconvenience and price via station and consumer subsidies }

\author{Lingling Shi | Suresh P. Sethi | Metin Çakanyıldırım}
\date{}


%New command to display footnote whose markers will always be hidden
\let\svthefootnote\thefootnote
\newcommand\blfootnotetext[1]{%
  \let\thefootnote\relax\footnote{#1}%
  \addtocounter{footnote}{-1}%
  \let\thefootnote\svthefootnote%
}

%Overriding the \footnotetext command to hide the marker if its value is `0`
\let\svfootnotetext\footnotetext
\renewcommand\footnotetext[2][?]{%
  \if\relax#1\relax%
    \ifnum\value{footnote}=0\blfootnotetext{#2}\else\svfootnotetext{#2}\fi%
  \else%
    \if?#1\ifnum\value{footnote}=0\blfootnotetext{#2}\else\svfootnotetext{#2}\fi%
    \else\svfootnotetext[#1]{#2}\fi%
  \fi
}

\newunicodechar{ı}{\ifmmode\imath\else{$\imath$}\fi}
\newunicodechar{€}{\ifmmode\text{\euro}\else\euro\fi}

\begin{document}
\maketitle
Naveen Jindal School of Management, University of Texas at Dallas, Richardson, Texas, USA

\section*{Correspondence}
Lingling Shi, Naveen Jindal School of Management, University of Texas at Dallas, Richardson, TX 75080, USA.\\
Email: \href{mailto:lingling.shi@utdallas.edu}{lingling.shi@utdallas.edu}

Handling Editor: Chris Tang and Subodha Kumar

\begin{abstract}
Environmental and energy independence concerns lead to government subsidies for electric vehicles (EVs). Operational decisions for a government are (i) to incentivize EV ownership by a direct consumer subsidy, a station subsidy that reduces charging inconvenience, or by both subsidies; and (ii) to minimize subsidy expenditure or to maximize EV adoption level. We model the interactions between the government and the charging supplier as a Stackelberg game and study the optimal structure of subsidies by incorporating charging inconvenience. We prove that this inconvenience is decreasing convex in the number of stations. In the expenditure minimization case, the optimal policy depends on the government adoption target and the charging station construction cost. If the adoption target is below a threshold that depends on the construction cost, the government provides pure consumer subsidy or no subsidy; otherwise, a combination of consumer and station subsidies is optimal. As the construction cost increases, the charger builds fewer stations, regardless of the subsidy type. We establish that expenditure minimization and adoption maximization yield the same subsidy policy if the charging inconvenience is linear. In a real-life case, we find numerically that a station subsidy alone is optimal if the construction cost is not low but the adoption target is low. Besides, a long driving range reduces the need for subsidies significantly if the construction cost is high, whereas a long charging time necessitates high expenditure allocated mostly to a station subsidy.
\end{abstract}

\section*{KEYWORDS}
government subsidies, convex charging inconvenience, electric vehicles, sustainable operations

\section*{1 | INTRODUCTION}
Climate change is fueled by the conventional combustion vehicles (CVs) that, for example, account for $28 \%$ of total greenhouse gas emissions in the United States (EPA, 2018). Moreover, the transportation sector is responsible for the largest share of U.S. petroleum consumption at approximately $70 \%$ (EIA, 2018). The answer to reducing both greenhouse gas emissions and oil dependence seems to lie in electric vehicles (EVs). Yet for all its advantages, the current adoption levels are much lower than the government targets. For example, the EV market share was around $4.5 \%$ in China, $2.5 \%$ in Europe, and $2.5 \%$ in the United States by 2018. According to IEA (2020), the EV30@30 campaign sets the collective market share goal of 10 major countries

\footnotetext{Accepted by Chris Tang and Subodha Kumar, after one revision.
}
representing two-thirds of EV sales worldwide at $30 \%$ by 2030 .

The main barriers to EV adoption include high purchase price and range anxiety (e.g., Carley et al., 2013; Egbue \& Long, 2012). The retail price of 2019 Nissan EV in the United States, after deducting the tax credit, is still $25 \%$ higher than that of a comparable Nissan CV (\href{http://nissanusa.com}{nissanusa.com}).

Range anxiety refers to "worry on the part of a person driving an electric car that the battery will run out of power before the destination or a suitable charging point is reached" according to the Oxford Online Dictionary (\href{http://lexico.com}{lexico.com}). Thus, it involves the frequency of charging (depending on the miles that the charged battery lasts) as well as the density of charging stations and length of charging time. Although the EV range has increased recently, it is still limited compared to that of CVs. Whereas the EV range and the charging time depend on the battery technology, the density (number\\
\includegraphics[max width=\textwidth, center]{2025_07_10_4f2749288484428c6552g-02}

FIGURE 1 Subsidies to the charger and consumers [Color figure can be viewed at \href{http://wileyonlinelibrary.com}{wileyonlinelibrary.com}]\\
per area) of charging stations can be increased with appropriate incentives. This density is an important positive signal to potential EV buyers. As in Sperling (2018), a visible network of charging stations can convince potential buyers that they will experience limited charging inconvenience. Several hundred thousand public charging stations are to be installed in the United States to serve twenty million EVs by 2030 (Brattle, 2020). However, without mass EV sales, the construction of charging stations is hardly profitable (e.g., Schroeder \& Traber, 2012). Taking this into account, potential EV buyers anticipate a significant charging inconvenience and often postpone their purchases. If more potential buyers purchase EVs, the demand for charging stations and in turn the density of these stations will increase. This is a positive feedback loop referred to as the "chicken-and-egg" dilemma (Ma et al., 2019): more buyers induce more charging stations which induces more buyers! To jump-start and sustain this loop, the government can offer subsidies to a charging supplier (charger) and/or to EV buyers (consumers); see Figure 1.

IEA (2020) reports that environmental and sustainability objectives drive governments worldwide to support EV developments. Many countries are maintaining or extending their current policies, for example, Germany increased the consumer subsidy in 2020. Subsidy types vary among governments; some use subsidies to reduce the EV purchase price, while others reduce the charging station construction cost. A federal subsidy program in the United States grants a tax credit of $\$ 7500$ for consumers. It is now a part of the Inflation Reduction Act of 2022 (The White House, 2022). The Norwegian government is the driver of charging infrastructure with an investment of $€ 6$ million in 2009 (ICCT, 2017). Singapore spent $\$ 20$ million in setting up an extensive charging station network and in providing subsidies towards the purchase of EVs (APEC, 2017). Sierzchula et al. (2014) and Langbroek et al. (2016) show a strong positive relation between EV adoption and subsidies. Unfortunately, most of the extant literature discusses one subsidy in isolation, such as the consumer subsidy (Chemama et al., 2019; Cohen et al., 2016). Silvia and Krause (2016) show empirically that a combined subsidy policy outperforms a pure policy, but without analyzing the optimal combination of subsidies. Since reducing purchase price or supporting charging stations interact with each other, it is of considerable interest to explore the expenditure of different combinations of subsidies and their impact on EV adoption. We aim to study this issue analytically.

We formulate a Stackelberg game between the government (she) and the charging supplier (charger, he) to obtain the\\
optimal (equilibrium level of) subsidies. Our study is the first to incorporate the response of the charger-a critical player in the EV ecosystem. We address a few important questions. First, what is the optimal subsidy structure, that is, a consumer subsidy only, a station subsidy only, or a combination? Second, what is the optimal number of charging stations? Third, how do the optimal subsidies change with the charger's station construction cost and the government's adoption target? Intuitively, we would expect more charging stations in response to an increase in the station subsidy. We would also expect that the station subsidy rises in the construction cost. These expectations do not always hold as detailed below.

\section*{1.1 | Literature survey}
A large number of studies in energy and environment fields focus on EV adoption. Sierzchula et al. (2014) find that charging infrastructure, financial incentives, and the local presence of production facilities are positively correlated with EV adoption. Studying the charging events in the United States and the United Kingdom, Neaimeh et al. (2017) show that public charging stations play a key role in improving EV adoption and suggest incentives to accelerate the growth of charging networks. In a German case study, Schroeder and Traber (2012) analyze the economic rationale of public charging stations and find that providing charging service is hardly profitable in the early stage. A similar analysis is performed with the Seattle travel data by Dong et al. (2014). Studying the impact of consumer anxieties, Lim et al. (2015) show that improved charging infrastructure typically yields higher adoption. Abouee-Mehrizi et al. (2021) show that deploying EVs in the car-sharing market is optimal only if their charging speed, their range, and number of charging stations are high enough. Describing the history and future of EVs, Sperling (2018) argues that they are inevitable as well as desirable for sustainable transportation and emphasizes that government subsidies are essential for quick adoption. These identify the important features relating to EV adoption and point out charging infrastructure and government subsidies as important factors.

The growing literature on sustainable operations (Kleindorfer et al., 2005) includes a stream of research on subsidy programs related to our work. Reviewing 180 papers published between 1992 and 2014, Joglekar et al. (2016) emphasize the importance of jointly studying operational decisions with policy considerations. Under a high level of environmental concern, Krass et al. (2013) show that subsidizing green technology while taxing emissions improves green technology adoption and social welfare. Baker and Solak (2014) develop a multimodal approach to analyze energy technology investment policies that address climate change issues. Raz and Ovchinnikov (2015) study coordination between the government and the supplier of public goods to achieve maximal welfare through rebates and/or subsidies. Berenguer et al. (2017) model a pricesetting newsvendor to study the effect of subsidies on for-profit and not-for-profit firms. Yu et al. (2018) study the\\
conditions under which the government should subsidize consumers only, manufacturers only, or both, motivated by a home appliance subsidy program in China. Taylor and Xiao (2019) analyze the impact of a distribution channel and consumer awareness on the donor's optimal subsidy design. Bai et al. (2021) study government subsidy programs that incentivize consumer trade-ins of used products. In the context of subsidy termination, Shi et al. (2022) analyze the government's and green product manufacturers' decisions under individual/group learning. Generally, these papers have studied the impact of a pure subsidy policy. Those considering combined policies (e.g., Raz \& Ovchinnikov, 2015), merely analyze the effectiveness of subsidies in reducing the cost or price of a product, similar to the consumer subsidy in our context but offered to consumers or firms. So these combined policies are in fact combinations of the same type of subsidy to different recipients. By contrast, we study two different types of subsidies, that is, the consumer subsidy to reduce EV price and the station subsidy to induce more stations for reducing range anxiety.

The booming smart city movement worldwide stimulates related operations management research. Qi and Shen (2019) view electrified mobility and infrastructures as important opportunities to expand the research scope. Mak et al. (2013) develop models of battery-swapping station planning and conduct numerical examples to study the impacts of battery standardization and technology development. In the same context of battery-swapping stations, Avci et al. (2015) study the effectiveness of reducing carbon emissions and oil dependence. Inspired by a swappable battery design, Shi and Hu (2022) study a flexible EV battery lease program with temporarily up/downgrading options. Parker et al. (2019) discuss the current operational and policy issues and future research opportunities in the electric power industry. Given EV fleets as virtual power plants, Kahlen et al. (2018) propose a mixed rental-trading strategy to study potential benefits of the fleet owner and consumers by exploiting battery capacity for rental and energy trading. Valogianni et al. (2020) design adaptive pricing to induce the desired EV charging demand profile. While scheduling charging operations, Wu et al. (2021) cleverly consider delayed charging to lower the emissions and cost of electricity.

Combinations of subsidies are attracting increasing attention in industry and academia. According to IEA (2020), EV subsidies "are evolving to a more holistic policy portfolio," and major countries have begun to provide subsidies for both the purchase of an EV and the expansion of charging networks. Ma et al. (2019) study the optimal structure of a product subsidy and a service infrastructure subsidy for an integrated firm (e.g., Tesla) providing products and the service infrastructure. The infrastructure has a binary nature, either deployed or not, so it can be roughly related to stations that are deployed incrementally. They suggest that it is optimal to only subsidize the product when the deployment cost is high. However, our optimal policy under a high construction cost may be a combination of subsidies or a single subsidy depending on the adoption target. In addition, we\\
individually consider the charger, rather than integrating it with the EV manufacturer, to model the charging suppliers (e.g., ChargePoint, Blink, EVgo) that are financially independent of the manufacturers. ChargePoint has a $58 \%$ share of the charging station market in the United States according to Alternative Fuels Data (DOE, 2021). Moreover, we capture the characteristics of the chicken-and-egg dilemma (Brozynski \& Leibowicz, 2022) in the EV ecosystem more precisely by modeling the charging inconvenience as a function of the number of charging stations.

To find the optimal number of charging stations, Yu et al. (2022) independently study a coordination problem, where the government builds charging stations directly and/or provides a per-station subsidy. In contrast, we have the government set only a subsidy policy to induce an independent charging supplier to build stations. Such scenarios exist in practice and need to be studied, they remarkably also lead to different results in the structure of optimal consumer and station subsidies. Whereas Yu et al. (2022) never offer both a station subsidy and a consumer subsidy, we show conditions under which a combination of consumer and station subsidies is optimal. Also, under these conditions, restricting to offer only pure subsidy policies can be quite suboptimal. Indeed, illustrated numerically when the adoption target is high, the optimal expenditure when only a pure consumer subsidy can be offered can be twice that when a combined subsidy policy is optimal. Yu et al. (2022) show that a station subsidy always increases in the construction cost, while we find it first increases then decreases as the construction cost increases. Other modeling differences between Yu et al. (2022) and this paper are that their demand is linear in the number of charging stations and their objective is to minimize the government's expenditure, whereas our demand incorporating charging inconvenience is quadratic to capture the realistic feature of marginally diminishing returns, and we maximize adoption in addition to minimizing subsidy expenditure.

Two recent papers that analyze (green technology) subsidies in the context of EVs are closely related to ours. Cohen et al. (2016) focus on the consumer subsidy with a pricesetting supplier. Chemama et al. (2019) consider the effect of fixed and flexible consumer subsidies offered to a supplier. These papers hint at two related government subsidy problem formulations: minimize the subsidy expenditure to ensure an adoption target and maximize the adoption level subject to a given budget. But they only qualitatively argue that the two problems are similar and only solve the former one. Naming the former problem as the budget-driven problem and the latter as the adoption-driven problem, we rigorously compare them. While they consider only a consumer subsidy, we consider a station subsidy jointly with a consumer subsidy. Expanding the charging station network, the station subsidy lowers the range anxiety, so it should be realistically an integral part of a subsidy model formulation. Even so, to the best of our knowledge, none of the existing papers integrate a station subsidy with a consumer subsidy while considering the impact of charging inconvenience, and our model fills this gap in the literature.

\section*{1.2 | Contributions to literature and managerial insights}
First, our analysis captures the impact of charging inconvenience-an important consequence and impediment of EV ownership. We demonstrate that the charging inconvenience is decreasing convex in the number of stations. The linear charging inconvenience is a special case, that is likely to happen in a symmetric road network. However, nonsymmetric networks are common in practice. We find that quadratic inconvenience fits the real-life data of Beijing better than linear inconvenience does. We study both linear and nonlinear inconveniences. Also, we compare the effectiveness of subsidizing the charger and/or consumers and find that the government provides pure consumer subsidy if the adoption target is below a threshold that depends on the construction cost; otherwise, a combined policy of consumer and station subsidies is optimal.

Second, we consider an unrestricted budget-driven problem to provide a more complete view of the optimal policy structure. Although the budget-driven problem with nonnegative subsidies is widespread in practice, the unrestricted problem achieves a better objective value than does the budget-driven problem under the same adoption target. We solve both problems. For high construction costs, the optimal structures of the two problems are the same. For low construction costs and adoption levels, however, the solution of the budget-driven problem is more complex than that of the unrestricted problem. This is in part because the station subsidy of the unrestricted problem is negative when the adoption target is less than a threshold.

Third, we prove that the budget-driven and adoptiondriven problems yield precisely the same subsidy policy. Specifically, under linear inconvenience, the solution of the budget-driven problem can also be obtained from the adoption-driven problem whose budget is limited by the minimum expenditure of the budget-driven problem. In reverse, the solution of the adoption-driven problem can be obtained via that of the budget-driven problem. Towards obtaining the same solution from these two problems, we provide a pairing equation that relates the budget to the adoption level. Upon incorporating an EV manufacturer as a price setter into our model, we find that the optimal decisions of the charging supplier and the government under the manufacturer's optimal EV price are structurally the same as those under an exogenous EV price.

In addition, we provide several managerial insights to different parties. As expected, the optimal subsidies depend on the adoption target and the station construction cost. The charger builds more stations when the construction cost decreases or the adoption target increases. However, he does not always build more stations as the station subsidy increases. A noteworthy observation in the budget-driven problem under low construction cost and quadratic inconvenience is that he interestingly builds fewer stations in response to an increase in the station subsidy. We find\\
that the station subsidy only partially compensates for the construction cost increase and that the spread between the construction cost and the station subsidy increases in the construction cost.

From the government's perspective, we find the optimal consumer and station subsidies and identify some interesting properties. As the construction cost increases, one would expect a higher station subsidy. This is not always the case. The station subsidy is zero at first, then becomes positive, and finally goes back to zero. Given a very high construction cost, she finds it too expensive to subsidize the charger and simply gives up. Under a high construction cost, we observe that the threshold, over which a station subsidy is offered, rises in the construction cost. By contrast, the threshold in a low construction cost case drops sso that the government's willingness to provide a station subsidy grows in the construction cost. On the other hand, the consumer subsidy always increases with the construction cost. The government increases the consumer subsidy to induce the demand decreased by fewer stations as the construction cost increases. Hence, consumer and station subsidies interact with each other and need to be studied jointly.

We conduct a numerical study with real-life data. From the fueling data in Beijing, we estimate the charging inconvenience and verify that it is decreasing and mostly convex in the number of stations. We compute the optimal decisions, and obtain the magnitude of subsidies and show their consistency with the analytical results. Furthermore, we numerically analyze different policy combinations to show their impact on the optimal expenditures. We find that the expenditure of pure consumer subsidy can be twice that of the combined policy. The pure station subsidy is more effective than the pure consumer subsidy if the adoption target is less than a threshold; otherwise, the pure station subsidy is infeasible. We also investigate the subsidies as the driving range increases and the charging time decreases. As the driving range increases, the charging frequency decreases, and in turn, the charging inconvenience decreases. Consequently, the government spends less under a longer driving range. When we consider the cost of charging time, the charging inconvenience increases. Both consumer and station subsidies increase in this case, while a larger portion of the increased expenditure is spent on the station subsidy.

We obtain closed-form optimal subsidies under quadratic inconvenience by using a linear demand formulation and an assumption of early adopters. The linear demand formulation is well-accepted in the literature. Besbes and Zeevi (2015) show that the widely used linear demand is sufficient to obtain good decisions. The assumption of early adopters says that some consumers buy EVs with the current number of stations and no subsidies. This is a weak assumption and is currently satisfied, but it was perhaps not so a century ago when EVs were first introduced and had the cost of approximately three times as much as CVs (DOE, 2014). So, with significantly high purchase prices and no subsidies, EVs disappeared from the market (Santini, 2011). Now with a better understanding\\
\includegraphics[max width=\textwidth, center]{2025_07_10_4f2749288484428c6552g-05}\\
of sustainability and the role of carbon emissions, governments are motivated to offer subsidies. Nevertheless, they are uncomfortable about the challenging task of choosing appropriate subsidy programs and of optimizing the chosen program for the best utilization of their limited financial resources. This paper aims to help governments with this task to avoid the mistakes of the last century in the developed countries and new mistakes in the developing countries.

The paper is organized as follows. In Section 2, we model the government and charging supplier decisions. In Section 3, we analyze the optimal subsidy structure under quadratic charging inconvenience. In Section 4, we investigate three extensions including one in which the manufacturer can decide the EV price. We provide numerical analyses in Section 5 and conclude the paper in Section 6. The proofs and complementary discussions are relegated to the appendices in the Supporting Information.

\section*{2 | DESCRIPTION OF MODELS}
We consider a Stackelberg game in which the government (she) is the leader and the charging supplier (charger, he) is the follower under government subsidies. The subsidies involve a consumer subsidy $r$ and a station subsidy $s$. A consumer subsidy is a price reduction for each EV purchased, while the station subsidy is a cost reduction for each charging station constructed. We consider subsidies that remain the same over a single season of a year or possibly longer. For example, the EV tax credit in the United States had remained the same since its launch in 2010 until the end of 2021. The sequence of events in our game is shown in Figure 2.

The government subsidizes a market whose participants are the charger and consumers. With an adoption target or a budget limit, she decides the type of subsidies and the amount to allocate for each type. Then, the charger decides the number of charging stations to build. All of these decisions are made before the sales season. Then a consumer decides to buy an EV or not. The government transfers a consumer subsidy $r$ (if any) per EV to consumers who purchase EVs and a station subsidy $s$ (if any) per station to the charger; recall Figure 1.

To concentrate on the effect of different subsidies, we consider a charger without capital constraints. The manufacturer is not a player in our main context, and EV price $p_{0}$ is exogenous. The government can estimate $p_{0}$ with a cost-plusmargin method before deciding on the subsidies. Chemama et al. (2019) also assume the price to be exogenous in their subsidy design problem. We extend our main model to incorporate the manufacturer's price decision in Section 4.3 and\\
find that the structure of the optimal subsidies is the same as that in the exogenous price case.

\section*{2.1 | Charging inconvenience}
We consider charging inconvenience as an additional cost assessed by potential buyers when purchasing an EV as opposed to a CV. This inconvenience comes from a short driving range, scarcity of charging stations, and a long charging time. New technologies increase the range and reduce the charging time; however, these usually take many years. One of the ways to mitigate charging inconvenience is to build more charging stations (e.g., Dong et al., 2014; Sierzchula et al., 2014). Hence, we model charging inconvenience as a function of the number of stations under the current technology. We suppose that there are $n_{0}$ charging stations at the beginning of the season. Then the charging inconvenience with $n$ additional charging stations is $h\left(n_{0}+n\right)$. The charging inconvenience is zero when the number of stations is greater than a threshold, say $n_{g}$. Then the charging inconvenience is $h\left(n_{0}+n\right)>0$ for $n_{0}+n<n_{g}$ and $h\left(n_{g}\right)=0$.

Dong et al. (2014) consider missed trips as an inconvenience and show that it is decreasing convex in the number of stations. Bernstein et al. (2021) use the term supply congestion that causes the inconvenience of waiting for a taxi and assume it to be an increasing convex function of the utilization rate (demand divided by supply). This waiting time (inconvenience), when considered as a function of supply only, becomes a decreasing convex function. Similarly, the charging inconvenience, when measured in terms of average traveled distance, can be expected to be convex in the number of charging stations. We obtain this property in the flowing lemma.

Lemma 1. The reduction in the average distance traveled to the closest station caused by an additional station decreases with the number of stations.

This property is based on finding the change in the average distance when an additional station is located by greedily minimizing this distance while keeping the locations of the existing stations fixed. Because of this incremental aspect, our station location problem differs from the classical simultaneous location problem that involves locating all facilities simultaneously. Our incremental location problem represents the practice better, but its optimal distance cannot be less than that of the simultaneous location problem. While the ratio of the optimal distances in the two problems and the incrementtal location algorithm to improve this ratio have been studied\\
\includegraphics[max width=\textwidth, center]{2025_07_10_4f2749288484428c6552g-06}

FIGURE 3 Charging inconvenience induced by travel as the number of stations rises [Color figure can be viewed at \href{http://wileyonlinelibrary.com}{wileyonlinelibrary.com}]\\
(e.g., Mettu \& Plaxton, 2003), the optimal average distance in the incremental location problem (Lemma 1) has not been studied, to the best of our knowledge. The lemma as applied in our context means that the more charging stations there are, the less is the charging inconvenience, and this inconvenience is marginally diminishing in the number of charging stations.

The charging inconvenience based on the traveled distance in Lemma 1 is linear if and only if every additional station causes the same amount of saving in the traveled distance. This may happen in completely symmetric transportation networks. Networks in practice have plenty of asymmetries, so it is important to consider a nonlinear charging inconvenience. Based on the real-life travel data in Beijing in Figure 3 (elaborated later in Section 5), we find that the $R$-squared value of the quadratic regression model is $94 \%$, while it is just $79 \%$ for the linear model. That is, quadratic inconvenience better represents the traveled distance than linear inconvenience does.

A consumer buys an EV if its effective price is no more than the willingness-to-pay, which can capture the price of the alternative CV and the premium attached to the environmentally friendly EV. The effective price is the actual price paid after subtracting the consumer subsidy and adding the cost of charging inconvenience over the EV's lifetime, that is, the consumer considers the total cost of ownership. Hereafter, we set the time horizon to 1 year and convert the revenues, costs, and subsidies to a corresponding annual amount over the lifetime-referred to as the annualized values. We assume all parameters and functions are common knowledge and list them in Table 1.

\section*{2.2 | Charger's problem}
Potential consumers have the intention to buy a vehicle and make up the market. Their total demand $Q(p)$ decreases in the price $p$. The effective price of an EV is $p_{0}-r+h\left(n_{0}+\right.$ $n$ ). Since $p_{0}$ is an exogenous variable, we can write the EV demand in terms of ( $n, r$ ) as


\begin{equation*}
d(n, r)=Q\left(p_{0}-r+h\left(n_{0}+n\right)\right) \text { for } n \geq 0 \tag{1}
\end{equation*}


TABLE 1 Important notation

\begin{center}
\begin{tabular}{|l|l|}
\hline
\multicolumn{2}{|c|}{Input parameters and functions} \\
\hline
$c$ & Construction cost of a charging station \\
\hline
$\rho$ & Charging profit (revenue minus the charging service cost) per EV collected by the charger \\
\hline
$p_{0}$ & Price of an EV charged by the manufacturer \\
\hline
$Q(p)$ & Demand at the effective price of $p$ \\
\hline
$h(m)$ & \begin{tabular}{l}
Charging inconvenience with $m$ charging stations \\
Decision variables \\
\end{tabular} \\
\hline
$n$ & Additional number of charging stations constructed by the charger \\
\hline
$r$ & Consumer subsidy provided by the government for each EV purchased \\
\hline
$s$ & Station subsidy provided by the government for each charging station constructed \\
\hline
\end{tabular}
\end{center}

Since $Q(\cdot)$ is a decreasing function and $h\left(n_{0}+n\right)$ decreases in $n, d(n, r)$ increases in $n$.

The charger (he) maximizes his profit by deciding the number of charging stations. He invests in building stations and collects revenue from charging EVs. The key trade-off is between the additional profit generated from adding charging stations and the construction cost of these stations. For simplicity, we consider no capacity constraint on a charging station, for example, as in Avci et al. (2015). Since we use the profit $\rho$, that is, after deducting the cost for charging service, the charger's problem is to maximize his annualized profit


\begin{equation*}
\max _{n \geq 0}\{\pi(n)=\rho d(n, r)-(c-s) n\} \tag{2}
\end{equation*}


We use $n(r, s)$ to denote the number of (additional) stations constructed in response to ( $r, s$ ).

In practice, the number of charging stations tends to increase with the consumer and station subsidies. This is why governments offer subsidies to expand the charging network. To mimic this increasing relationship in our model, we obtain sufficient conditions on $Q$ and $h$ to guarantee that $n(r, s)$ increases in $r$ and $s$.

Lemma 2. If (i) $Q$ is linear or (ii) $Q$ is convex, $\left|h^{\prime}(\cdot)\right| \leq 1$ and $\left|h^{\prime \prime}(\cdot) / h^{\prime}(\cdot)\right| \geq\left|Q^{\prime \prime}(\cdot) / Q^{\prime}(\cdot)\right|$, we have

\begin{itemize}
  \item The $E V$ demand $d(n, r)$ is concave in $n$.
  \item The charger's best response $n(r, s)$ increases in the consumer subsidy $r$ and station subsidy $s$.
\end{itemize}

A part of Condition (ii) is on the slope of charging inconvenience $h$ and can be satisfied by increasing the monetary units used to measure $h$. The other part is on the relative curvatures defined as the ratio $\left|f^{\prime \prime}(\cdot) / f^{\prime}(\cdot)\right|$ for a function $f$. Such a condition occurs in the utility theory where the relative curvature of a convex utility function represents the coefficient of the absolute risk aversion (Mas-Colell et al., 1995). The concavity result of $d(n, r)$ in $n$ follows the well-established\\
law of diminishing returns (Shephard \& Färe, 1974). That is the marginal increase in demand caused by adding charging stations decreases in the number of stations. The next section is for $n(r, s)$ functions increasing in $r$ and $s$, for which sufficient conditions are presented in (i) or (ii). We use only (i) in Section 3 and afterward.

\section*{2.3 | Government's problem}
The government introduces subsidies to achieve a predetermined EV adoption target $A \geq 0$. For example, the United States set the adoption target of one million EVs by 2015 (DOE, 2011). The budget-driven model of minimizing the total subsidy expenditure to achieve $A$ is formulated as follows:

Budget-driven model: $\quad \min _{r \geq 0, s \geq 0}\{r d(n(r, s), r)+s n(r, s):$


\begin{equation*}
d(n(r, s), r) \geq A\} \tag{3}
\end{equation*}


As both the objective and constraint functions increase in $r$ and $s$, the constraint in (3) is binding at the optimal solution when the adoption target $A$ is above the lowest achievable demand $d(n(0 ; 0) ; 0)$. Therefore, we only consider the equality constraint $d(n(r, s), r)=A$ in Section 3. As is common, we assume a linear demand $Q(p)=(a-b p)^{+}$for $a, b \geq 0$, where $(x)^{+}=\max \{x, 0\}$ for a number $x$. The EV demand (1) induced by the number of stations and consumer subsidy is


\begin{align*}
d(n, r)= & a-b\left(p_{0}-r+h\left(n_{0}+n\right)\right) \text { for } r \geq-\left(a-b\left(p_{0}\right.\right. \\
& \left.\left.+h\left(n_{0}+n\right)\right)\right) / b \tag{4}
\end{align*}


Early adopter assumption implies that there are consumers who will purchase EVs with $n_{0}$ charging stations and no government subsidies, that is, $a-b\left(p_{0}+h\left(n_{0}\right)\right)>0$. Since $n \geq 0$ and $h$ decreases, this assumption implies $a-b\left(p_{0}+h\left(n_{0}+\right.\right.$ $n)) \geq 0$. We check that the optimal solutions without the positive demand constraint $r \geq-\left(a-b\left(p_{0}+h\left(n_{0}+n\right)\right)\right) / b$ satisfy this constraint and remain optimal. Note that the only nonlinear part in $d(n, r)$ is $h\left(n_{0}+n\right)$, which is decreasing convex in $n$ as shown in Section 2.1. So, $d(n, r)$ is increasing concave in $n$, indicating a diminishing rate of return to demand from the number of charging stations. Moreover, the charger's best response $n(r, s)$ can be shown to depend only on $s$ in the linear demand model and the notation of $n(s)$ is used from now on instead of $n(r, s)$.

\section*{3 | OPTIMAL SUBSIDIES UNDER QUADRATIC CHARGING INCONVENIENCE}
The charging inconvenience in a region is related to that region's road network; specifically, its functional form\\
depends on the topology of the network. Satellite towns developed after careful planning are likely to have symmetric road networks (star or grid topology) and a linear charging inconvenience (at least for a range of station numbers), as the marginal contribution of one additional charging station at a particular location is close to those of the other stations that have been installed in locations symmetric to the particular one in the same network. However, coastal cities or historic cities (e.g., Beijing in our numerical example) have asymmetric road networks and therefore nonlinear charging inconveniences that tend to be convex. For these cities, quadratic forms are more appropriate. Therefore, we analyze the optimal subsidies under a quadratic inconvenience in this section. Later, we visit linear inconveniences.

A quadratic charging inconvenience can be expressed as $h\left(n_{0}+n\right)=\left[\beta-\alpha n+\lambda n^{2}\right]^{+}$, where $\beta$ denotes the maximum charging inconvenience for $n=0$. As the charging inconvenience is decreasing in the number of stations, we consider $n \leq \alpha /(2 \lambda)$. The charging inconvenience is nonnegative. Note $h\left(n_{0}+n\right) \geq 0$ for any $n$ and $\lambda \geq \alpha^{2} /(4 \beta)$, whereas $h\left(n_{0}+n\right)=0$ at $n=\left(\alpha-\sqrt{\alpha^{2}-4 \lambda \beta}\right) /(2 \lambda)$ for $\lambda<\alpha^{2} /(4 \beta)$. The case of $\lambda \geq \alpha^{2} /(4 \beta)$ has a higher quadratic coefficient compared to the case of $\lambda<\alpha^{2} /(4 \beta)$. So the former and latter are, respectively, referred to as (strongly) quadratic and weakly quadratic cases. The weakly quadratic case is between the quadratic case and the linear case. It can be similarly studied (EC. 1 in the Supporting Information) as the quadratic case analyzed here. In the case of quadratic charging inconvenience, the number of stations is $n \leq \alpha /(2 \lambda)$ and the EV demand is obtained by inserting $h\left(n_{0}+n\right)$ into (4):


\begin{align*}
d(n, r)= & a-b\left(p_{0}-r+\beta-\alpha n+\lambda n^{2}\right) \text { for } r \geq-\left(a-b\left(p_{0}\right.\right. \\
& \left.\left.+\beta-\alpha n+\lambda n^{2}\right)\right) / b \tag{5}
\end{align*}


Specialization of the charger's problem (2) for the EV demand (5) yields the optimal number of stations (see EC. 1 in the Supporting Information):


\begin{equation*}
n_{q}(s)=\min \left\{(\alpha-(c-s) /(b \rho))^{+}, \alpha\right\} /(2 \lambda) . \tag{6}
\end{equation*}


This is the charger's best response to the government's station subsidy. It is independent of the consumer subsidy. But it increases in the station subsidy, that is, the charger builds more stations when provided with a higher $s$. The total expenditure is increasing in $s$ and (6) implies $s \leq c$.

With no consumer subsidy, let the demands under maximum and no charging inconveniences, respectively, be $A_{\beta}$ and $A_{0}$, and $A_{0} \geq A_{\beta} \geq 0$ following from the early adopter assumption. Specifically,


\begin{equation*}
A_{\beta}=a-b p_{0}-b \beta \text { and } A_{0}=a-b p_{0} . \tag{7}
\end{equation*}


We indicate quadratic inconvenience with subscript " $q$ " and denote the optimal values as $n_{q}, r_{q}$, and $s_{q}$. In addition, the\\
\includegraphics[max width=\textwidth, center]{2025_07_10_4f2749288484428c6552g-08}

FIGURE 4 Solutions under quadratic inconvenience with no restriction on the subsidies [Color figure can be viewed at \href{http://wileyonlinelibrary.com}{wileyonlinelibrary.com}]\\
superscript " $u$ " in $n_{q}^{u}, r_{q}^{u}$, and $s_{q}^{u}$ indicates the unrestricted setting below.

\section*{3.1 | No restriction on the subsidies}
We first consider an unrestricted (in sign) subsidy setting where the government may provide a positive or negative subsidy if it is in her best interest. This unrestricted problem describes a complete view in the sense of yielding a potentially better objective value compared to its restricted version with nonnegative subsidies only. As the Stackelberg leader, the government knows the charger's best response (6). Then, the unrestricted budget-driven problem becomes


\begin{align*}
\text { Unrestricted budget-driven model: } & \min _{r, s}\left\{r d\left(n_{q}(s), r\right)+s n_{q}(s):\right. \\
& \left.d\left(n_{q}(s), r\right)=A ; s \leq c\right\} \tag{8}
\end{align*}


where $\quad d\left(n_{q}(s), r\right)=a-b\left(p_{0}-r+\lambda n_{q}(s)^{2}-\alpha n_{q}(s)+\beta\right)$ from (5) and $s \leq c$ comes from (6).

Proposition 1. The optimal solution of the unrestricted model in (8) is QB1: $n_{q}^{u}=0, s_{q}^{u}=0, r_{q}^{u}=$ $\left(A-A_{\beta}\right) / b$ if $A \leq L_{q}^{u}(c)=c / \alpha-b \rho$; otherwise, the optimal is $Q B 2$ : $n_{q}^{u}=(\alpha-(c+\alpha b \rho) /(A+2 b \rho)) /(2 \lambda), \quad s_{q}^{u}=$ $\left((A+b \rho) c-\alpha b^{2} \rho^{2}\right) /(A+2 b \rho), r_{q}^{u}=\left(A-A_{\beta}+b\left(\lambda\left(n_{q}^{u}\right)^{2}-\right.\right.$ $\left.\left.\alpha n_{q}^{u}\right)\right) / b$.

The optimal solution of the unrestricted model in (8) is shown in Figure 4, where the optimal policy structure depends on both the adoption target and construction cost. To better understand the optimal structure, we consider two government policies: a pure policy of providing either a consumer subsidy or a station subsidy and a combined policy of providing both subsidy types. If the adoption target is higher than the threshold $L_{q}^{u}(c)$, the government is better off providing a combined policy; otherwise, she uses a pure consumer subsidy. Because the amount of station subsidy under a low adoption target is insufficient to cover a high construction cost, the charger builds no station. With this in mind, the government provides only a consumer subsidy.

On the other hand, she provides the station subsidy when the construction cost is low or the adoption target is high. In summary, she supports the EV market with at least one of the subsidies.

Corollary 1. The optimal station subsidy of (8) is nonnegative if the adoption target is higher than a threshold, that is, $A \geq \max \left\{\alpha b^{2} \rho^{2} / c-b \rho, 0\right\}$.

When the construction cost is high ( $c \geq \alpha b \rho$ ), the threshold in Corollary 1 is zero and the station subsidy is always nonnegative. When $c<\alpha b \rho$ however, the government may provide a positive or negative station subsidy, depending on the adoption target $A$. In particular, the station subsidy is nonnegative if $A$ is higher than $\alpha b^{2} \rho^{2} / c-b \rho$; otherwise, it is negative. We observe that this adoption threshold decreases with the construction cost. In turn, the government's willingness to provide a nonnegative station subsidy grows as $c$ increases. In summary, the government introduces a nonnegative subsidy to compensate for a high construction cost. It increases this subsidy at the rate the construction cost increases when the adoption target is very high; when the adoption level is close to zero, it increases the subsidy at half the rate of the cost increase.

\section*{3.2 | Nonnegative restriction on both subsidies}
Governments in practice tend to provide nonnegative subsidies to promote EV development. To model this, the consumer and station subsidies are restricted to be nonnegative. Viewing the unrestricted model in (8) as a relaxation of the corresponding model, the optimal expenditure of (8) can be smaller than that of the budget-driven model with the same adoption target. The budget-driven model under quadratic inconvenience with nonnegative restriction on the consumer and station subsidies is given by (3) for $d\left(n_{q}(s), r\right)=a-$ $b\left(p_{0}-r+\lambda n_{q}(s)^{2}-\alpha n_{q}(s)+\beta\right)$ and $s \leq c$.

Naturally, we use the solution of the unrestricted problem (8) to the extent it applies to the budget-driven problem. The consumer subsidy must be negative if $A<A_{\beta}$, so we focus on the adoption target $A \geq A_{\beta}$. To describe the optimal solution of (3) in the next proposition, we define two adoption thresholds $L_{q}(c)$ and $G_{q}(c ; A)$.


\begin{align*}
L_{q}(c)= & \left\{\begin{array}{lll}
\alpha b^{2} \rho^{2} / c-b \rho & \text { if } & c<\alpha b \rho \\
c / \alpha-b \rho & \text { if } & c \geq \alpha b \rho
\end{array}\right\} . \\
G_{q}(c ; A)= & \left\{\begin{array}{lll}
A_{\beta}+\frac{b}{4 \lambda}\left(\alpha^{2}-\frac{c^{2}}{b^{2} \rho^{2}}\right) & \text { if } & c \leq \frac{\alpha b^{2} \rho^{2}}{A+b \rho} \\
A_{\beta}+\frac{b}{4 \lambda}\left(\alpha^{2}-\frac{(c+\alpha b \rho)^{2}}{(A+2 b \rho)^{2}}\right) & \text { if } & \frac{\alpha b^{2} \rho^{2}}{A+b \rho}<c<\alpha A_{\beta}+\alpha b \rho \\
A_{\beta} & \text { if } & c \geq \alpha A_{\beta}+\alpha b \rho
\end{array}\right\} . \tag{9}
\end{align*}


\begin{center}
\includegraphics[max width=\textwidth]{2025_07_10_4f2749288484428c6552g-09}
\end{center}

FIGURE 5 Solutions of budget-driven problem. $A<A_{\beta}$ is infeasible under nonnegative subsidies [Color figure can be viewed at \href{http://wileyonlinelibrary.com}{wileyonlinelibrary.com}]\\
$L_{q}(c)$ resembles the threshold in Corollary 1 and is used below to identify the instances $A \leq L_{q}(c)$ of low adoption requiring no station subsidy. It first decreases hyperbolically then increases linearly in the construction cost. $G_{q}(c ; A)$ specifies the instances $A \leq G_{q}(c ; A)$ requiring no consumer subsidy.

Proposition 2. When $A \geq G_{q}(c ; A)$, the optimal solution ( $n_{q}, s_{q}, r_{q}$ ) of the budget-driven model (3) is given for the low construction cost case (LCC) and high construction cost case (HCC):

\begin{itemize}
  \item LCC: $c \leq \alpha b \rho$. The optimal solution is QB3: $n_{q}=(\alpha-$ $c /(b \rho)) /(2 \lambda), s_{q}=0, r_{q}=\left(A-A_{\beta}+b\left(\lambda\left(n_{q}\right)^{2}-\alpha n_{q}\right)\right) / b$ if $A \leq L_{q}(c)$; otherwise, the optimal is $Q B 2$.
  \item $H C C: c>\alpha b \rho$. The optimal solution is $Q B 1$ if $A \leq L_{q}(c)$; otherwise, the optimal is QB2.
\end{itemize}

Moreover, the optimal subsidies are $s_{q}=0$ and $r_{q}=0$ if $A_{\beta} \leq$ $A \leq G_{q}(c ; A)$ and $c \leq \alpha b^{2} \rho^{2} /(A+b \rho)$.

The optimal solution in Proposition 2 is shown in Figure 5, where "+" and "0," respectively, mean that the corresponding subsidy is positive and zero. For $A \geq G_{q}(c ; A)$, the government should use a pure consumer subsidy under LCC and HCC if $A \leq L_{q}(c)$; otherwise, the combined policy is optimal. If $A<G_{q}(c ; A)$ and $c \leq \alpha b^{2} \rho^{2} /(A+b \rho)$ (i.e., $A \leq L_{q}(c)=$ $\alpha b^{2} \rho^{2} / c-b \rho$ under LCC), the optimal subsidies of (3) is zero and, in turn, the minimum expenditure becomes zero. This is because $\left(r_{q}, s_{q}\right)=(0,0)$ attains the adoption targets below $G_{q}(c ; A)$. Under the remaining adoption level $A \in$ $\left[A_{\beta}, G_{q}(c ; A)\right)$ and the construction cost $c \in\left(\alpha b^{2} \rho^{2} /(A+\right.$ $b \rho), \alpha A_{\beta}+\alpha b \rho$ ), the optimal solution of (3) is more subtle and can be numerically computed.

Under HCC, the charger builds no station without a station subsidy, whereas under LCC he builds a few (but not as many as the maximum) without a station subsidy. The charger builds the maximum number of stations and the inconvenience becomes zero, if the construction cost $c$ is extremely low, in which case the government provides only consumer\\
subsidy or no subsidy. If $c$ is extremely high, the charger builds no station and the inconvenience remains unchanged, which in turn motivates the government to provide only a consumer subsidy.

In Figure 5, if we fix $A$ and consider rising $c$ with a horizontal move, the number of charging stations always decreases in $c$. The station subsidy is zero at first, then becomes positive, and finally goes back to zero as $c$ increases. For a very high $c$, it is impossible to achieve any adoption target through station subsidy, so the government provides pure consumer subsidy. A noteworthy observation is that when the government provides a higher station subsidy to compensate for a higher construction cost, the charger may choose to build fewer stations. For example, she provides some station subsidy in QB2 unlike no subsidy in QB3, but the charger builds fewer stations in QB2 than in QB3 as $(\alpha-(c+\alpha b \rho) /(A+$ $2 b \rho)) /(2 \lambda)<(\alpha-c /(b \rho)) /(2 \lambda)$. This is because the charger, in addition to the station subsidy provided, considers the construction cost, in particular, the spread $c-s_{q}$. We find that the spread increases in $c$, which means that the government partially compensates for the increase in $c$. If we fix $c$ and consider increasing $A$ with a vertical move in Figure 5, the station subsidy increases, and in turn the number of charging stations increases.

The solutions of the budget-driven model and its unrestricted counterpart differ mainly under LCC as the government may want to adopt a negative station subsidy under low construction costs. If the government commits to nonnegative subsidies, she converts a negative subsidy to no subsidy in the budget-driven model for $A \leq L_{q}(c)$ and ends up with a station subsidy that first increases and then decreases in $c$. Under LCC, high adoption targets $A>L_{q}(c)$ can be optimally attained only with a station subsidy. Furthermore, this threshold $L_{q}(c)$ decreases under LCC. That is the government's willingness to provide a station subsidy increases so as to mitigate the decrease in the number of charging stations as $c$ increases. Recall that quadratic inconvenience has a diminishing return rate which decreases even faster at a smaller number of stations. With linear inconvenience (i.e., constant rate of return), we indeed show later that a low construction cost under LCC may suffice to attain any adoption level $A$ without a station subsidy.

\section*{4 | EXTENSIONS}
In this section, we investigate three extensions: (i) the adoption-driven problem and its relation to the budget-driven problem, (ii) a government that maximizes total social welfare, and (iii) the manufacturer's price is a decision variable.

\section*{4.1 | Adoption-driven model and its relation to budget-driven model}
An alternative to the budget-driven problem is the adoption-driven model, where the government is\\
\includegraphics[max width=\textwidth, center]{2025_07_10_4f2749288484428c6552g-10}

FIGURE 6 Solutions under linear inconvenience. Left: adoption-driven problem. Right: budget-driven problem [Color figure can be viewed at \href{http://wileyonlinelibrary.com}{wileyonlinelibrary.com}]\\
interested in maximizing the adoption-level subject to a given budget $B$.


\begin{align*}
& \text { Adoption-driven model: } \max _{r \geq 0, s \geq 0}\{d(n(r, s), r): r d(n(r, s), r) \\
& \quad+\operatorname{sn}(r, s) \leq B\} . \tag{10}
\end{align*}


For example, China grants a budget of RMB 100 billion to support the development of the new energy vehicle industry from 2011 to 2020 (APEC, 2017).

As discussed earlier, linear inconvenience is an important special case and it leads to an approximation and several insights. So we provide the optimal subsidies of (10) under linear inconvenience. We denote the optimal values of the number of stations, consumer subsidy, and station subsidy as $n_{l}, r_{l}$, and $s_{l}$, and continue to refer, respectively, to $c \leq \alpha b \rho$ and $c>\alpha b \rho$ as LCC and HCC.

Proposition 3. The optimal subsidies of the adoption-driven model (10) under linear charging inconvenience are given below for two cases by using the budget threshold

\[
\Theta(c)= \begin{cases}\beta(c-\alpha b \rho) / \alpha & \text { if } \alpha b \rho<c \leq \alpha A_{0}+\alpha b \rho  \tag{11}\\ (c-\alpha b \rho)\left((c-\alpha b \rho) / \alpha-A_{\beta}\right) /(\alpha b) & \text { if } c>\alpha A_{0}+\alpha b \rho\end{cases}
\]

\begin{itemize}
  \item LCC: The optimal subsidies are $s_{l}=0, r_{l}=\left(-A_{0}+\right.$ $\left.\sqrt{A_{0}^{2}+4 b B}\right) /(2 b)$.
  \item HCC: If $B<\Theta(c)$, then $s_{l}=0, \quad r_{l}=\left(-A_{\beta}+\right.$ $\left.\sqrt{A_{\beta}^{2}+4 b B}\right) /(2 b) ; \quad$ otherwise $, \quad s_{l}=c-\alpha b \rho, \quad r_{l}=$ $\left(-A_{0}+\sqrt{A_{0}^{2}+4 b(B-\beta(c-\alpha b \rho) / \alpha)}\right) /(2 b)$.
\end{itemize}

The optimal solution of the adoption-driven model (10) is depicted on the left panel of Figure 6. Under LCC, it is optimal for the government to provide pure consumer subsidy no matter what her budget is. Because the charger builds as many stations as possible anyway. Under HCC, she is better off providing combined subsidies if the budget is higher than the threshold $\Theta(c)$; otherwise, she provides pure consumer\\
subsidy. Because the amount of the station subsidy under a low budget is not enough to cover the high construction cost, the charger does not build any station. With this in mind, the government provides only a consumer subsidy. On the other hand, if the budget is sufficient to cover the high construction cost, she provides a station subsidy to induce the charger to build the maximum number of stations. Note that both $s_{l}=(c-\alpha b \rho)^{+}$and $\Theta(c)$ increase in $c$. As the construction cost increases, we find that the number of instances that have a positive station subsidy goes down, whereas the amount of station subsidy if offered goes up.

We observe that the adoption-driven model (10) is obtained by exchanging the positions of the objective of the budgetdriven model (3) with the left-hand side of the constraint of (3). It is not surprising to find some similarity between the solutions of these models, at least in terms of $n_{l}$ values and positivity of subsidies; see Propositions 3 and 4 and two panels of Figure 6. Figures 5 and 6 depict the optimal subsidies under quadratic and linear inconveniences. Both of these solutions show that the government should use a pure consumer subsidy if the adoption target is low and a combined subsidy policy otherwise.

Proposition 4. The optimal solutions of the budget-driven model (3) under linear charging inconvenience are given below for two cases by using the adoption threshold

\[
L(c)= \begin{cases}\left(A_{\beta}+\sqrt{A_{\beta}^{2}+4 b \beta(c-\alpha b \rho) / \alpha}\right) / 2 & \text { if } \alpha b \rho<c \leq \alpha A_{0}+\alpha b \rho,  \tag{12}\\ (c-\alpha b \rho) / \alpha & \text { if } c>\alpha A_{0}+\alpha b \rho .\end{cases}
\]

\begin{itemize}
  \item LCC: The optimal solutions are $s_{l}=0, r_{l}=\left(A-A_{0}\right)^{+} / b$, and $n_{l}=\beta / \alpha$.
  \item HCC: If $A<L(c)$, then $s_{l}=0, r_{l}=\left(A-A_{\beta}\right) / b$, and $n_{l}=$ 0 ; otherwise, $s_{l}=c-\alpha b \rho, r_{l}=\left(A-A_{0}\right)^{+} / b$, and $n_{l}=$ $\beta / \alpha$.
\end{itemize}

Encouraged by the similarity between the solutions of the adoption-driven and the budget-driven models, we check if these solutions can be made numerically the same. For this\\
purpose, we relate the adoption target $A$ and the budget $B$ through the following equation:

Pairing equation:

\[
A=\left\{\begin{array}{cl}
\left.A_{0}+\sqrt{A_{0}^{2}+4 b B}\right) / 2 & \text { in } L C C  \tag{13}\\
\left.A_{0}+\sqrt{A_{0}^{2}+4 b B-4 b \beta(c \alpha-b \rho)}\right) / 2 & \text { if } B \geq \Theta(c) \text { in HCC } \\
\left.A_{\beta}+\sqrt{A_{\beta}^{2}+4 b B}\right) / 2 & \text { if } B<\Theta(c) \text { in HCC }
\end{array}\right\} .
\]

Taking its inverse, it can alternatively be written as

\[
B=\left\{\begin{align*}
A\left(A-A_{0}\right) / b & \text { in } L C C  \tag{14}\\
A\left(A-A_{0}\right) / b+\beta(c-\alpha b \rho) / \alpha & \text { if } A \geq L(c) \text { in } \mathrm{HCC} \\
A\left(A-A_{\beta}\right) / b & \text { if } A<L(c) \text { in } \mathrm{HCC}
\end{align*}\right\} .
\]

Derivation and equivalence of these equations are in the proof of the next corollary.

Corollary 2. The budget-driven model with a given adoption $A$ and the adoption-driven model with a given budget $B$ yield the same optimal solution if $A$ and $B$ satisfy the pairing equation (13).

The highest attainable adoption level is the market size $a$, so $A \leq a$ in the right panel of Figure 6. Under LCC, any adoption level can be attained with no station subsidy; in particular, the adoption level of $a$ is obtained with consumer subsidy $p_{0}$, station subsidy 0 , and $\alpha / \beta$ stations. The corresponding total expenditure $a p_{0}$ can also be obtained by inserting $A=a$ in $A\left(A-A_{0}\right) / b$. The adoption level of $a$ under HCC is given by substituting $a$ for $A$ in $A\left(A-A_{0}\right) / b+\beta(c-\alpha b \rho) / \alpha$ if $A \geq L(c)$, otherwise it is $A\left(A-A_{\beta}\right) / b$. The corresponding expenditures are $a p_{0}+\beta(c-\alpha b \rho) / \alpha$ and $a\left(p_{0}+\beta\right)$, respectively. The budget increases from 0 to $a p_{0}$ under LCC and to $\min \left\{a p_{0}+\right.$ $\left.\beta(c-\alpha b \rho) / \alpha, a\left(p_{0}+\beta\right)\right\}$ under HCC, as the adoption level increases from $A_{\beta}$ to $a$; see Figure 6. Over these ranges, a pair of adoption and budget value that satisfies (13) yields precisely the same subsidy policy from the budget-driven and adoption-driven models. Hence, the government captures all of the optimal policies even after ignoring one of these models.

\section*{4.2 | Welfare maximization with an extension of the budget-driven model}
In welfare economics, a Benthamite welfare function sums the utility of each individual in the society to obtain overall\\
\includegraphics[max width=\textwidth, center]{2025_07_10_4f2749288484428c6552g-11}

FIGURE 7 Solution of the welfare-driven problem under linear inconvenience [Color figure can be viewed at \href{http://wileyonlinelibrary.com}{wileyonlinelibrary.com}]\\
welfare. Then the government maximizes this social welfare, measured in our context by the sum of the charger's profit and the consumer surplus net of the government's expenditure. The consumer surplus is given by the difference between the consumer willingness to pay and the market price. In particular, the consumer surplus is $d^{2}(n(s), r) /(2 b)$ in the linear demand model. Thus, the total welfare is


\begin{align*}
W(r, s)= & {\left[\rho d(n(s), r)-(c-s) n(s)+d^{2}(n(s), r) /(2 b)\right] } \\
& -[r d(n(s), r)+\operatorname{sn}(s)] \tag{15}
\end{align*}


The optimal number of stations and subsidies are denoted by $n_{w}, r_{w}, s_{w}$ for the following model:

Welfare-driven model: $\max _{r \geq 0, s \geq 0}\{W(r, s): d(n(s), r) \geq A\}$.

The adoption constraint in (16) may be not binding at the optimal solution for low levels of adoption and construction cost. A case in point is $A<A_{0}+b \rho$ under LCC, where the welfare objective rises in $r$ when the adoption target is lower than the threshold $A<A_{0}+b \rho$ and the optimal is at the subsidy that achieves this threshold. The complete solution of the welfare-driven problem is given in the following proposition and shown in Figure 7:

Proposition 5. The optimal subsidies of the welfare-driven model (16) under linear charging inconvenience are given below by using the construction cost threshold


\begin{align*}
& M_{w}(A) \\
& = \begin{cases}\alpha\left(A_{0}+b \rho-b \beta / 2\right) & \text { for } A<A_{\beta}+b \rho \\
\alpha\left(\left(A_{0}+b \rho\right)^{2}+A\left(A-2 A_{\beta}\right)-2 b \rho A\right) /(2 b \beta) & \text { for } A_{\beta}+b \rho \leq A<A_{0}+b \rho \\
\alpha A & \text { for } A \geq A_{0}+b \rho\end{cases} \tag{17}
\end{align*}


\begin{itemize}
  \item LCC: For $A<A_{0}+b \rho$, we have $s_{w}=0, r_{w}=\rho$; otherwise, $s_{w}=0, r_{w}=\left(A-A_{0}\right) / b$.
  \item HCC: (i) For $A<A_{\beta}+b \rho$, we have $s_{w}=c-\alpha b \rho, r_{w}=\rho$ if $c \leq M_{w}(A)$; otherwise, $s_{w}=0, r_{w}=\rho$; (ii) For $A_{\beta}+$ $b \rho \leq A<A_{0}+b \rho$, we have $s_{w}=c-\alpha b \rho, r_{w}=\rho$ if $c \leq$ $M_{w}(A)$; otherwise, $s_{w}=0, r_{w}=\left(A-A_{\beta}\right) / b$; (iii) For $A \geq$ $A_{0}+b \rho$, we have $s_{w}=c-\alpha b \rho, r_{w}=\left(A-A_{0}\right) / b$ if $c \leq$ $M_{w}(A)$; otherwise, $s_{w}=0, r_{w}=\left(A-A_{\beta}\right) / b$.
\end{itemize}

In Figure 7, the threshold $L_{w}(c)$ is induced by $M_{w}(A)$; it indeed is the inverse of $M_{w}$ when $c>\alpha\left(A_{0}+b \rho-b \beta / 2\right)$. Under LCC, the government provides no station subsidy at any adoption level, as the charger builds the maximum number of stations anyway. Under HCC, she provides combined subsidies if the construction cost is lower than $M_{w}(A)$; otherwise, she provides pure consumer subsidy. From Figures 6 and 7 , the optimal structure for the welfare-driven problem is qualitatively the same as that for the budget-driven problem, if $A \geq A_{\beta}+b \rho$ and $c \geq \alpha b \rho$. Furthermore, the threshold $L(c)$ is lower than $L_{w}(c)$ when $c \geq \alpha\left(A_{0}+b \rho\right)$, and they are parallel to each other with the same slope of $\alpha^{-1}$. For a construction cost higher than $\alpha\left(A_{0}+b \rho\right)$, the government is less willing to subsidize the charger when she considers the total welfare, compared to when she considers only the expenditure.

\section*{4.3 | EV pricing by the manufacturer under quadratic and linear inconveniences}
Incorporating an EV manufacturer as a player that decides the EV price, the budget-driven model of Section 3.1 is extended. In the case of manufacturer pricing, it is interesting to investigate the optimal decisions of the government and the charging supplier. Initially, we consider a manufacturer that sets the price $p$ per EV before the charging supplier sets the number $n$ of charging stations. The government sets the subsidies by considering the responses from the manufacturer and the charging supplier. For notational brevity, we normalize the cost per EV to be zero. The manufacturer maximizes the profit from selling EVs by choosing the price in the manufacturer's problem


\begin{equation*}
\max _{p \geq 0}\left\{\pi_{m}(p)=p d(n(r, s, p), r)\right\} \tag{18}
\end{equation*}


Here $n(r, s, p)$ is the optimal number of stations for the EV price $p$ and subsidies $(r, s)$.

To obtain the optimal solution, we proceed from the charger's problem to the manufacturer's problem and then to the government's problem. We first solve the charger's problem (2) under quadratic charging inconvenience to obtain the optimal number of stations. EV price is fixed when solving the charger's problem, which is analyzed as before; $n(r, s, p)=n_{q}(s)$ in (6). Inserting the charger's response into the manufacturer's problem (18) yields the optimal EV\\
price


\begin{equation*}
p_{m}\left(n_{q}(s), r\right)=\left(a+b r-b h\left(n_{q}(s)\right)\right) /(2 b) \tag{19}
\end{equation*}


The charger responds only to the station subsidy with the number of stations, whereas the manufacturer responds to the consumer and station subsidies with the EV price. Given the responses for the number of stations and EV price, we eventually find the optimal solution ( $s_{m}, r_{m}$ ) for the government's unrestricted budget-driven model (8). The subscript $m$ denotes the manufacturer pricing case. With no consumer subsidy, the demands under maximum and no charging inconveniences in the case of manufacturer pricing are denoted by $A_{\beta}^{m}$ and $A_{0}^{m}$ with $A_{0}^{m} \geq A_{\beta}^{m} \geq 0$.


\begin{align*}
& A_{\beta}^{m}=a-b((a-b \beta) /(2 b)+\beta)=(a-b \beta) / 2 \text { and } \\
& A_{0}^{m}=a-a / 2=a / 2 \tag{20}
\end{align*}


We report the solution of the government's model under endogenous EV price below as a corollary to Proposition 1. Since the optimal number of stations is unaffected by the EV price $p$, the optimal subsidies remain the same even if the charging supplier moves before the manufacturer. This points to the robustness of subsidies concerning the sequence of manufacturer and charger decisions.

Corollary 3. The optimal solution of the unrestricted budget-driven model under quadratic inconvenience in the case of manufacturer pricing is QM1: $n_{m}=0, s_{m}=$ $0, r_{m}=2\left(A-A_{\beta}^{m}\right) / b$ and $p_{m}=A / b$ if $A \leq c / \alpha-b \rho$; otherwise, the optimal is QM2: $n_{m}=(\alpha-(c+\alpha b \rho) /(A+$ $2 b \rho)) /(2 \lambda), \quad s_{m}=\left((A+b \rho) c-\alpha b^{2} \rho^{2}\right) /(A+2 b \rho), \quad r_{m}=$ $\left(2\left(A-A_{\beta}^{m}\right)+b\left(\lambda\left(n_{m}\right)^{2}-\alpha n_{m}\right)\right) / b$ and $p_{m}=A / b$.

The structure of the optimal subsidies in the case of manufacturer pricing is the same as that in Proposition 1. The manufacturer's optimal price is $A / b$, which is similar to that in Proposition 4 in Yu et al. (2022). The manufacturer can charge a higher price when the government targets a higher adoption level. This is because a higher adoption target leads to a higher consumer subsidy, which counterbalances the increases in the EV price. Predicting this in advance and attempting to avoid it, the government would ask the manufacturer to fix the EV price in the absence of subsidies and particularly before subsidies are announced. Such an initiative of the government would make the EV price fixed at $p_{0}$ as in Section 2-3.

We conclude this section by providing the results also under linear convenience as a corollary of Proposition 4. The structures of the optimal subsidies with exogenous and endogenous pricing once again remain the same.

Corollary 4. The optimal solutions of the budget-driven model under linear charging inconvenience in the case of

TABLE 2 Parameter estimates in the base case

\begin{center}
\begin{tabular}{llllllll}
\hline
Parameter & $\boldsymbol{\lambda}$ & $\boldsymbol{\alpha}$ & $\boldsymbol{\beta}$ & $\boldsymbol{p}_{\boldsymbol{0}}$ & $\boldsymbol{\rho}$ & $\boldsymbol{a}$ & $\boldsymbol{b}$ \\
\hline
Estimated value & 0.017 & 20.1 & 5500 & 35,600 & 180 & $1,369,370$ & 32.86 \\
\hline
\end{tabular}
\end{center}

manufacturer pricing are given below by using the adoption threshold


\begin{align*}
& L_{m}(c)= \\
& \quad \begin{cases}\left(A_{\beta}^{m}+\sqrt{\left(A_{\beta}^{m}\right)^{2}+2 b \beta(c-\alpha b \rho) / \alpha}\right) / 2 & \text { if } \quad \alpha b \rho<c \leq \alpha A_{0}^{m}+\alpha b \rho \\
(c-\alpha b \rho) / \alpha & \text { if } \quad c>\alpha A_{0}^{m}+\alpha b \rho\end{cases} \tag{21}
\end{align*}


\begin{itemize}
  \item LCC: The optimal subsidies are $s_{l}^{m}=0, r_{l}^{m}=2(A$ $\left.A_{0}^{m}\right)^{+} / b$. The optimal EV price is $p_{l}^{m}=\max \left\{A_{0}^{m}, A\right\} / b$. The optimal number of charging stations is $n_{l}^{m}=\beta / \alpha$.
  \item HCC: If $A<L_{m}(c)$, then $s_{l}^{m}=0, r_{l}^{m}=2\left(A-A_{\beta}^{m}\right) / b, p_{l}^{m}=$ $A / b$ and $n_{m}=0$; otherwise, $s_{l}^{m}=c-\alpha b \rho, r_{l}^{m}=2(A-$ $\left.A_{0}^{m}\right)^{+} / b, p_{l}^{m}=\max \left\{A_{0}^{m}, A\right\} / b$ and $n_{l}^{m}=\beta / \alpha$.
\end{itemize}

\section*{5 | NUMERICAL ILLUSTRATIONS}
In this section, we introduce a base case and other similar problems, solve them, and comment on the impacts of subsidy types, driving range, and charging time.

\section*{5.1 | Base case}
We infer charging inconvenience from the average travel distance to the closest gas station. The reasons to use gas station data are that this market has reached maturity so the number of stations is sufficient for our study and that the consumer does not change driving habits so the travel distances to gas and charging stations should be structurally similar. We use the location data of 762 gas stations from Gaode Maps inside Beijing's $5^{\text {th }}$ Ring road, and divide the city into $5 \mathrm{~km} \times 5 \mathrm{~km}$ blocks with the demand of each block at its center. We calculate the distance between two locations based on the shortest path and use the average travel distance to the closest station over demand blocks to represent the fueling inconvenience. We assess the change of fueling inconvenience in the number of fueling stations in a reverse process. First, we start from the maximum number of stations and compute the inconvenience. Next, we remove one fueling station from a pair of stations closest to each other, compute the inconvenience after removing this station, and repeat this process until sufficient (number of station, inconvenience) pairs are obtained. The average travel distance, depicted in Figure 3, decreases in the number of stations. The intuitive\\
removal criterion of fueling stations does not necessarily minimize the inconvenience for a given number of stations and causes slight deviations in the figure from the convexity of Lemma 1.

We assume that an EV has a lifetime of 6 years. As in Section 2, we set the time horizon to be a year. That is, we consider annual demand in (4), annual profit in (2), and annual adoption level in (3). We first estimate the parameters in (4). The average EV price is about $\$ 35,600$, and the average CV price is $\$ 25,000$. In 2018, around 547,747 new vehicles were registered in Beijing, and, therefore, the demand equation is $547,747=a-25,000 b$. In addition, the elasticity of linear demand is $\epsilon(p)=-b p /(a-b p)$. From Table 18.1 in Van Ryzin (2012), the absolute value of the elasticity for automobiles is 1.5 in the short run. By solving the demand and elasticity equations, we obtain $a=$ $1,369,370$ and $b=32.86$. Another important parameter is the net charging revenue $\rho$ in (2). We consider the EV driving range as 150 miles with a battery capacity of 40 kWh , and the electricity price is $12 ¢ / \mathrm{kWh}$. So the driving cost is $40 * 12 / 150=3.2 ¢ /$ mile . In Avci et al. (2015), the annual average driving distance per vehicle is about 18,000 miles. We assume that $50 \%$ of the total electricity consumed by EVs is from charging stations and the profit margin is $50 \%$. Thus, the annual net charging revenue per EV is $180 * 3.2 * 0.5 *$ $0.5=\$ 144$. We round this up to $\$ 180$ to capture the charging and storage inefficiency and thus set $\rho=180$.

The only remaining parameters are in the quadratic inconvenience expression. We start with a regression on the data of Figure 3, that is, $y=0.00018 x^{2}-0.21931 x+67.53614$. We decrease the constant to 60 to capture the weakly quadratic case of $\lambda<\alpha^{2} /(4 \beta)$ while keeping the curvature. We construct the base case of the maximum inconvenience ( $n=0$ ) as follows. Assume that an EV is recharged once in 10 days on average, that is, $365 * 10 \%=36.5$ times annually. To come up with the maximum inconvenience, we consider the driver has to drive a long distance for charging, and this driving cost is $\$ 25$ each time. In this case, the maximum inconvenience is $\$ 5500 \approx 36.5 * 6 * \$ 25$ over 6 years; therefore, the charging inconvenience in $n$ is $h\left(n_{0}+n\right)=0.017 n^{2}-20.1 n+5500$ after scaling the regression outcome to obtain $\beta=5500$. We summarize parameter values in Table 2, where the first four parameters are for the lifetime and the others are for a year.

\section*{5.2 | Numerical solutions}
As the solutions of linear and strongly quadratic inconveniences are studied above, we only consider a weakly quadratic inconvenience here. We compute the optimal decisions for the budget-driven problem (3) with parameters in Table 2. The solution for weakly quadratic inconvenience has three construction cost cases (LCC, MCC (medium construction cost), HCC) that are separated by the construction cost thresholds $b \rho \sqrt{\alpha^{2}-4 \lambda \beta}=32,402$,\\
\includegraphics[max width=\textwidth, center]{2025_07_10_4f2749288484428c6552g-14}

FIGURE 8 Optimal values for the base case as a function of the adoption level [Color figure can be viewed at \href{http://wileyonlinelibrary.com}{wileyonlinelibrary.com}]\\
$\alpha b \rho=118,887$, and the specific adoption levels are $A_{0}=a-$ $b p_{0} \approx 200,000, A_{\beta}=a-b p_{0}-b \beta \approx 19,000$. The adoption levels in Model (3) are higher than $A_{\beta}$. So we plot the optimal expenditures, the consumer, and station subsidies in Figure 8 for $A \in[19000,300000]$ under three construction cost cases. We select a large construction cost of $\$ 600,000 /$ year to illustrate the optimal consumer subsidy differences under HCC and LCC/MCC. The construction costs under LCC and MCC are, respectively, $\$ 20,000$ and $\$ 60,000$ per year.

In Figure 8a, the optimal expenditure is increasing in $A$. In addition, a higher construction cost leads to a higher expenditure under the same adoption target. Figure 8b shows the optimal consumer subsidies under LCC, MCC, and HCC, respectively, while the optimal station subsidies are in Figure 8c. We draw two vertical scales in Figure 8c, where LCC and MCC use the left scale and HCC the right scale. The optimal subsidies here under LCC are exactly as predicted by the optimal solution for the weakly quadratic inconvenience (EC. 1 in the Supporting Information). In particular, the optimal consumer subsidy is zero if $A \leq 200,000$ and positive otherwise, and the optimal station subsidy is always zero. Under MCC, the optimal consumer subsidy in Figure 8b is the same as that under LCC, while the optimal station subsidy in Figure 8c is zero if $A<165,000$, then increases in $A$ and stays stable around 27,000 if $A>200,000$. The HCC case in Figure 8 shows that the government provides only consumer subsidy if $A \leq 24,000$, only station subsidy if $24,000<$\\
$A<200,000$, and both subsidies if $A \geq 200,000$. Under HCC, the threshold $L_{q}(c)=c / \alpha-b \rho$ is approximately 24,000 . If $A_{\beta}<A \leq L_{q}(c)$ under HCC, the optimal policy in weakly quadratic inconvenience (analyzed in EC. 1 in the Supporting Information) is a pure consumer subsidy, which coincides with the numerical results for $A \leq 24,000$.

In this numerical example, we have $p_{0}=35,600<a / b-$ $\rho\left(\alpha / \sqrt{\alpha^{2}-4 \lambda \beta}-1\right)=41,192$ and the construction cost threshold $\left(A_{0}+2 b \rho\right) \sqrt{\alpha^{2}-4 \lambda \beta}=1,039,101$. Therefore, $G_{q}(c ; A)$ is $A_{0}=200,000$ as both MCC and HCC thresholds are less than $1,039,101$. We check the optimal values for $A \geq 200,000$ in this example and see that the numerical results in Figure 8 turn out to be the same as those for $A \geq A_{0}$ predicted by the optimal solution for the weakly quadratic inconvenience. When $L_{q}(c)<A<A_{0}$ in MCC and HCC cases, we have no closed-form solutions and thus present the numerical results here. Under MCC, the government provides no subsidy if the adoption level is below the threshold (around 165,000 ) and provides pure station subsidy if the adoption level is between 165,000 and 200,000 . This is because the charger builds a few stations ( $n \approx 293$ ) if $s=0$, and in turn the adoption level is 164,330 without any subsidy. Under HCC, she provides pure station subsidy for $24,000<A<200,000$. The consumer subsidy increases under LCC and MCC. However, under HCC, the consumer subsidy in Figure 8b increases at first, then decreases, and finally increases in $A$. We also observe in Figure 8c that the station subsidy always increases.\\
\includegraphics[max width=\textwidth, center]{2025_07_10_4f2749288484428c6552g-15}

FIGURE 9 Impact of subsidy combination on expenditures. Left: MCC case; right: HCC case [Color figure can be viewed at \href{http://wileyonlinelibrary.com}{wileyonlinelibrary.com}]

\section*{5.3 | Impact of different subsidy types}
The government incurs substantive administrative costs to run subsidy programs. For example, CBO (2017) announced that annual federal administrative costs for the crop insurance program averaged $\$ 200$ million from 2010 to 2016, $2.3 \%$ of the total expenditure. These administrative costs must be taken into account, especially when implementing pure and combined policies. If the optimal solution is a pure policy (either consumer or station subsidy) without administrative cost, the inclusion of administrative costs does not alter the optimal policy. If, however, a combined subsidy program is optimal with no administrative cost, it ceases to be optimal if the administrative costs are higher than the difference between the expenditures of the combined subsidies and just its pure policy. The lower are the administrative costs, the lower are the odds for a pure policy to replace the combined policy. Thus, a government with e-governance initiatives for running subsidy programs can keep administrative costs low enough to institute a combined subsidy program.

It is of interest to evaluate how much is the expenditure difference between a pure policy and a combined policy. The optimal policy under LCC is pure consumer subsidy in Figure 8. So, we only illustrate the minimum expenditures under MCC and HCC here. From Figure 9, the optimal expenditure of pure consumer subsidy is always more than that of combined subsidies, and the difference becomes larger as the adoption level increases. The expenditure of pure consumer subsidy is twice that of combined subsidies when $A \geq 150,000$ under MCC and when $A \geq 50,000$ under HCC. The optimal expenditure of pure station subsidy is almost the same as that of the combined policy if $A \leq 200,000$. However, the adoption level under pure station subsidy cannot exceed $A_{0}=200,000$, and thus increasing station subsidy does not change the inconvenience, and nor, in turn, the adoption level. This suggests that pure station subsidy is as effective as combined subsidies only if the adoption target is not too high, that is, $A \leq A_{0}$. Otherwise, for very high adoption levels, the government has to use combined subsidies.

\section*{5.4 | Impact of the driving range}
As technology expands the driving range, the charging frequency decreases. We reduce the annual recharging times to 20 , and the maximum inconvenience becomes $\$ 3000=$ $20 * 6 * \$ 25$ over the lifetime. Therefore, the charging inconvenience is $h\left(n_{0}+n\right)=0.009 n^{2}-10.97 n+3000$ and $\lambda \leq$ $\alpha^{2} /(4 \beta)$ in this case, whereas all the other parameters remain the same as in Table 2. In this case, the construction cost thresholds are $b \rho \sqrt{\alpha^{2}-4 \lambda \beta}=20,778$ and $\alpha b \rho=64,885$, and the specific adoption levels are $A_{0}=a-b p_{0} \approx 200,000$ and $A_{\beta}=a-b p_{0}-b \beta \approx 101,000$. To compare with the base case, we plot the minimum expenditures for $A \in$ [19000,300000]. Since the optimal solution under LCC is pure consumer subsidy, we only present the expenditures under MCC and HCC.

In Figure 10, the expenditures under MCC are close to each other. Under HCC, however, the expenditure in the long-range case is less than that in the base case, and the difference is significant when $A<200,000$. In the interval of $101,000 \leq A<200,000$, the government provides pure station subsidy in the long-range case. Under each adoption level, the government can achieve the same reduction in charging inconvenience in the long-range case with fewer stations compared to the base case. By planning for fewer stations, therefore, she spends less when the range is longer.

\section*{5.5 | Impact of the charging time}
An EV takes a few hours to be fully charged, so we want to investigate the impact of the charging time. We consider consumer idle time-while waiting for charging-to be 20 min and the cost of time to be $\$ 6$ per hour. Thus the cost of time over 6 years is $\$ 500 \approx 36.5 * 6 * 1 / 3 * \$ 6$. Since this time does not depend on the number of stations, its impact on charging inconvenience can be captured with $\beta=$ $5500+500=6000$. All other parameters remain the same as in Table 2. We have $\alpha^{2}-4 \lambda \beta=-4<0$, which includes only\\
\includegraphics[max width=\textwidth, center]{2025_07_10_4f2749288484428c6552g-16(1)}

FIGURE 10 Impact of the driving range on expenditures. Left: MCC case; right: HCC case [Color figure can be viewed at \href{http://wileyonlinelibrary.com}{wileyonlinelibrary.com}]\\
\includegraphics[max width=\textwidth, center]{2025_07_10_4f2749288484428c6552g-16}

FIGURE 11 Impact of the charging time on expenditures [Color figure can be viewed at \href{http://wileyonlinelibrary.com}{wileyonlinelibrary.com}]

LCC and HCC cases in Proposition 2. Then, the construction cost threshold is $\alpha b \rho=118,887$, and the specific adoption levels are $A_{0} \approx 200,000$ and $A_{\beta} \approx 2394$. With only LCC and HCC cases when the charging time is considered, we compute the expenditures for the construction cost of $\$ 60,000$ or $\$ 600,000$ for $A \in[19000,300000]$.

In Figure 11, the expenditures with charging time are larger than those in the base case. When the charging time is considered, it increases both the consumer and station subsidies compared to those in the base case. Also, most of the increased expenditure-more than $99 \%$ in our settingis spent on the station subsidy. From the HCC case for $\lambda>$ $\alpha^{2} /(4 \beta)$, the threshold $L_{q}(c)=c / \alpha-b \rho$ is 24,000 , which is the same as that in the base case. When the adoption level is below this threshold, the government provides pure consumer subsidy in the charging time and base cases. With a lower $A_{\beta}$ in the charging time case, the government uses a larger consumer subsidy to achieve her adoption target. When $A<200,000$ under $c=60,000$ and $24,000<A<$ 200,000 under $c=600,000$, the government uses pure station subsidy.

The station subsidy in the charging time case is higher as the charging inconvenience incorporating the charging time is larger for a given number of stations. The charging time can be longer for larger batteries deployed for longer ranges\\
in Section 5.4, and so they lead to a higher station subsidy. Considering the charging time, the government plans for more stations, so the optimal expenditure is higher. To achieve her adoption target at a lower expenditure, the government can mandate the charger to provide an enhanced service environment (e.g., amenities) to reduce consumers' perceived waiting time (e.g., Fan et al., 2016).

\section*{6 | CONCLUSIONS AND FUTURE RESEARCH}
We develop analytical models of interactions between the government and the charging supplier to design effective subsidy programs incorporating charging inconvenience of EV (electric vehicle) owners. Specifically, we show how subsidies are impacted by the form of charging inconvenience, the charging station construction cost, and the government adoption target. The government provides a consumer subsidy and/or a station subsidy. The former reduces the effective price of an EV directly, and the latter does so indirectly through charging inconvenience. We analyze the optimal subsidy types and values and find that the optimal policy structure depends on the government adoption target and the station construction cost. Our results provide practical\\
insights also for the charging supplier. The charger should build as many stations as possible even with no subsidies when the construction cost is low. However, when the construction cost is high and the charger utilization rate is low at the beginning of EV market development, as is typical, the charger's decision strongly depends on the station subsidy from the government. While it should be clear that the charger builds fewer stations as the construction cost rises, we characterize this behavior precisely under different subsidy types.

The transportation sector is undergoing a major transformation driven by concerns over the environment, sustainability, technology advancement (NASEM, 2021a), and ensuring diversity and equity for underserved communities (NASEM, 2021b). To embrace and manage this transformation, there is an expanding need for analytical frameworks towards a deeper understanding of the initiatives by governments and industry. Several interesting directions exist for future studies. We treat a single season that may apply well to short horizons. In practice, subsidies may be phased out and new technologies develop in the long run. So a long-run subsidy formulation deploying a dynamic model would be of interest. The automobile industry is closely related to this transformation. The traditional automakers (e.g., GM) introduce EVs besides their current CVs, whereas new entrants (e.g., Tesla) only produce EVs. The optimal product portfolio and competitive strategy development are open issues. As in our paper, charging inconvenience is an important feature to incorporate in EV-related models. Infrastructure planning for battery charging and swapping is vital to stimulate EV adoption and efficient operation. The spatial and time-variant pricing schemes of charging services need to be investigated from the perspectives of the grid and vehicle owners. BofA (2021) observes that while the EV charging sector is growing fast, there is a dearth of studies comparing different business models such as an asset-light franchised station model (where a reseller partners with ChargePoint) and a charger owned and operated station model (e.g., Blink). It is more challenging to consider these problems under multiple types of batteries. The trade-off between investing in large capacity batteries and high density of stations is worth exploring. Towards sustainable transportation systems, additional research opportunities arise from the integration of business models and technology innovations in battery life, recycling, energy density, autonomous driving, distributed energy storage, as well as EV and grid interactions.

\section*{ACKNOWLEDGMENTS}
We highly appreciate the constructive and helpful comments from Chris Tang and an anonymous referee on an earlier draft of the paper. As a result, the paper is more complete and improved.

\section*{REFERENCES}
Abouee-Mehrizi, H., Baron, O., Berman, O., \& Chen, D. (2021). Adoption of electric vehicles in car sharing market. Production and Operations Management, 30(1), 190-209.

APEC, A. D. (2017). The impact of government policy on promoting new energy vehicles. Publication no. APEC\# 217-CT-01.3.\\
Avci, B., Girotra, K., \& Netessine, S. (2015). Electric vehicles with a battery switching station: Adoption and environmental impact. Management Science, 61(4), 772-794.\\
Bai, J., Hu, S., Gui, L., So, K. C., \& Ma, Z.-J. (2021). Optimal subsidy schemes and budget allocations for government-subsidized trade-in programs. Production and Operations Management, 30(8), 2689-2706.\\
Baker, E., \& Solak, S. (2014). Management of energy technology for sustainability: How to fund energy technology research and development. Production and Operations Management, 23(3), 348-365.\\
Berenguer, G., Feng, Q., Shanthikumar, J. G., \& Xu, L. (2017). The effects of subsidies on increasing consumption through for-profit and not-for-profit newsvendors. Production and Operations Management, 26(6), 1191-1206.\\
Bernstein, F., DeCroix, G. A., \& Keskin, N. B. (2021). Competition between two-sided platforms under demand and supply congestion effects. Manufacturing \& Service Operations Management, 23(5), 1043-1061.\\
Besbes, O., \& Zeevi, A. (2015). On the (surprising) sufficiency of linear models for dynamic pricing with demand learning. Management Science, 61(4), 723-739.\\
BofA (2021). EV charging and infrastructure. The EV charging download: Takeaways from our inaugural summit. Bank of America Global Research.\\
Brattle (2020). Getting to 20 million EVs by 2030: Opportunities for the electricity industry in preparing for an EV future. Brattle Group.\\
Brozynski, M. T., \& Leibowicz, B. D. (2022). A multi-level optimization model of infrastructure-dependent technology adoption: Overcoming the chicken-and-egg problem. European Journal of Operational Research, 300(2), 755-770.\\
Carley, S., Krause, R. M., Lane, B. W., \& Graham, J. D. (2013). Intent to purchase a plug-in electric vehicle: A survey of early impressions in large US cities. Transportation Research Part D: Transport and Environment, 18, 39-45.\\
CBO, C. B. O. (2017). Options to reduce the budgetary costs of the federal crop insurance. \href{https://www.cbo.gov/publication/53375}{https://www.cbo.gov/publication/53375}\\
Chemama, J., Cohen, M. C., Lobel, R., \& Perakis, G. (2019). Consumer subsidies with a strategic supplier: Commitment vs. flexibility. Management Science, 65(2), 681-713.\\
Cohen, M. C., Lobel, R., \& Perakis, G. (2016). The impact of demand uncertainty on consumer subsidies for green technology adoption. Management Science, 62(5), 1235-1258.\\
DOE. (2011). One million electric vehicles by 2015. \href{https://www1.eere}{https://www1.eere}. \href{http://energy.gov/vehiclesandfuels/pdfs/1_million_electric_vehicles_rpt.pdf}{energy.gov/vehiclesandfuels/pdfs/1\_million\_electric\_vehicles\_rpt.pdf}\\
DOE. (2014). The history of the electric car. \href{https://www.energy.gov/}{https://www.energy.gov/} articles/history-electric-car\\
DOE. (2021). Alternative fueling station locator by August 24, 2021. https:// \href{http://afdc.energy.gov/stations/#/analyze?fuel=elec}{afdc.energy.gov/stations/\#/analyze?fuel=elec}\\
Dong, J., Liu, C., \& Lin, Z. (2014). Charging infrastructure planning for promoting battery electric vehicles: An activity-based approach using multiday travel data. Transportation Research Part C: Emerging Technologies, 38, 44-55.\\
Egbue, O., \& Long, S. (2012). Barriers to widespread adoption of electric vehicles: An analysis of consumer attitudes and perceptions. Energy Policy, 48, 717-729.\\
EIA. (2018). Oil and petroleum products explained. \href{https://www.eia.gov/}{https://www.eia.gov/} energyexplained/oil-and-petroleum-products/use-of-oil.php\\
EPA. (2018). Fast facts on transportation greenhouse gas emissions. \href{https://www.epa.gov/greenvehicles/fast-facts-transportation-greenhouse-gas-emissions}{https://www.epa.gov/greenvehicles/fast-facts-transportation-greenhouse-gas-emissions}\\
Fan, Y., Guthrie, A., \& Levinson, D. (2016). Waiting time perceptions at transit stops and stations: Effects of basic amenities, gender, and security. Transportation Research Part A: Policy and Practice, 25(12), 1977-2001.\\
ICCT, W. P. (2017). Emerging best practices for electric vehicle charging infrastructure. \href{https://theicct.org}{https://theicct.org}\\
IEA. (2020). Global EV outlook. \href{https://www.iea.org}{https://www.iea.org}\\
Joglekar, N. R., Davies, J., \& Anderson, E. G. (2016). The role of industry studies and public policies in production and operations management. Production and Operations Management, 25(12), 1977-2001.

Kahlen, M. T., Ketter, W., \& van Dalen, J. (2018). Electric vehicle virtual power plant dilemma: Grid balancing versus customer mobility. Production and Operations Management, 27(11), 2054-2070.\\
Kleindorfer, P. R., Singhal, K., \& Van Wassenhove, L. N. (2005). Sustainable operations management. Production and Operations Management, 14(4), 482-492.\\
Krass, D., Nedorezov, T., \& Ovchinnikov, A. (2013). Environmental taxes and the choice of green technology. Production and Operations Management, 22(5), 1035-1055.\\
Langbroek, J. H., Franklin, J. P., \& Susilo, Y. O. (2016). The effect of policy incentives on electric vehicle adoption. Energy Policy, 94, 94-103.\\
Lim, M. K., Mak, H.-Y., \& Rong, Y. (2015). Toward mass adoption of electric vehicles: Impact of the range and resale anxieties. Manufacturing \& Service Operations Management, 17(1), 101-119.\\
Ma, G., Lim, M. K., Mak, H.-Y., \& Wan, Z. (2019). Promoting clean technology adoption: To subsidize products or service infrastructure? Service Science, 11(2), 75-95.\\
Mak, H.-Y., Rong, Y., \& Shen, Z.-J. M. (2013). Infrastructure planning for electric vehicles with battery swapping. Management Science, 59(7), 1557-1575.\\
Mas-Colell, A., Whinston, M., \& Green, J. (1995). Chapter 6: Choice under uncertainty. In Microeconomic theory. Oxford University Press, p. 190.\\
Mettu, R. R., \& Plaxton, C. G. (2003). The online median problem. SIAM Journal on Computing, 32(3), 816-832.\\
NASEM. (2021a). Investing in transportation resilience: A framework for informed choices. The National Academies Press. \href{https://doi.org/10}{https://doi.org/10}. 17226/26292\\
NASEM. (2021b). Resource guide for improving diversity and inclusion programs for the public transportation industry. The National Academies Press. \href{https://doi.org/10.17226/26230}{https://doi.org/10.17226/26230}\\
Neaimeh, M., Salisbury, S. D., Hill, G. A., Blythe, P. T., Scoffield, D. R., \& Francfort, J. E. (2017). Analysing the usage and evidencing the importance of fast chargers for the adoption of battery electric vehicles. Energy Policy, 108, 474-486.\\
Parker, G. G., Tan, B., \& Kazan, O. (2019). Electric power industry: Operational and public policy challenges and opportunities. Production and Operations Management, 28(11), 2738-2777.\\
Qi, W., \& Shen, Z.-J. M. (2019). A smart-city scope of operations management. Production and Operations Management, 28(2), 393406.

Raz, G., \& Ovchinnikov, A. (2015). Coordinating pricing and supply of public interest goods using government rebates and subsidies. IEEE Transactions on Engineering Management, 62(1), 65-79.\\
Santini, D. (2011). Electric vehicle waves of history: lessons learned about market deployment of electric vehicles. In S. Soylu (Ed.), Electric vehicles: The benefits and barriers (pp. 35-62). IntechOpen.\\
Schroeder, A., \& Traber, T. (2012). The economics of fast charging infrastructure for electric vehicles. Energy Policy, 43, 136-144.\\
Shephard, R. W., \& Färe, R. (1974). The law of diminishing returns. In W. Eichhorn, R. Henn, O. Opitz, \& R. W. Shephard (Eds.), Production theory (pp. 287-318). Springer.\\
Shi, L., Cakanyıldırım, M., \& Sethi, S. P. (2022). Dynamic pricing of green products under subsidy termination and coopetition (Working paper). the University of Texas at Dallas, Richardson, TX.

Shi, L., \& Hu, B. (2022). Battery as a service: Flexible electric vehicle battery leasing. SSRN 4082272. \href{https://doi.org/10.2139/ssrn}{https://doi.org/10.2139/ssrn}. 4082272\\
Sierzchula, W., Bakker, S., Maat, K., \& Van Wee, B. (2014). The influence of financial incentives and other socio-economic factors on electric vehicle adoption. Energy Policy, 68, 183-194.\\
Silvia, C., \& Krause, R. M. (2016). Assessing the impact of policy interventions on the adoption of plug-in electric vehicles: An agent-based model. Energy Policy, 96, 105-118.\\
Sperling, D. (2018). Electric vehicles: Approaching the tipping point. In D. Sperling, (Ed.), Three revolutions (pp. 21-54). Springer.\\
Taylor, T. A., \& Xiao, W. (2019). Donor product-subsidies to increase consumption: Implications of consumer awareness and profit-maximizing intermediaries. Production and Operations Management, 28(7), 17571772.

The White House. (2022). New OMB analysis: The Inflation Reduction Act will significantly cut the social costs of climate change. Available at \href{https://www.whitehouse.gov/omb/briefing-room/2022/08/23/}{https://www.whitehouse.gov/omb/briefing-room/2022/08/23/} (accessed date August 25, 2022)\\
Valogianni, K., Ketter, W., Collins, J., \& Zhdanov, D. (2020). Sustainable electric vehicle charging using adaptive pricing. Production and Operations Management, 29(6), 1550-1572.\\
Van Ryzin, G. (2012). Chapter 18: Models of demand. In: Ö. Özer \& R. Phillips (Eds.), The Oxford handbook of pricing management. Oxford University Press, p 506.\\
Wu, O. Q., Yücel, Ş., \& Zhou, Y. (2021). Smart charging of electric vehicles: An innovative business model for utility firms. Manufacturing \& Service Operations Management. Advance online publication. \href{https://doi.org/10}{https://doi.org/10}. 1287/msom.2021.1019\\
Yu, J. J., Tang, C. S., Li, M. K., \& Shen, Z.-J. M. (2022). Coordinating installation of electric vehicle charging stations between governments and automakers. Production and Operations Management, 31(2), 681696.

Yu, J. J., Tang, C. S., \& Shen, Z.-J. M. (2018). Improving consumer welfare and manufacturer profit via government subsidy programs: Subsidizing consumers or manufacturers? Manufacturing \& Service Operations Management, 20(4), 752-766.

\section*{SUPPORTING INFORMATION}
Additional supporting information can be found online in the Supporting Information section at the end of this article.

How to cite this article: Shi, L., Sethi, S. P., \& Çakanyıldırım, M. (2022). Promoting electric vehicles: Reducing charging inconvenience and price via station and consumer subsidies. Production and Operations Management, 31, 4333-4350. \href{https://doi.org/10.1111/poms}{https://doi.org/10.1111/poms}. 13871


\end{document}